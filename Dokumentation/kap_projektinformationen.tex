\chapter{Projektinformationen}


\section{Grundidee}
Bei schwingenden K�rpern entstehen musikalische T�ne. Diese K�rper k�nnen schwingende Saiten (Klavier, Gitarre, Violine), Membranen (Pauke, Trommeln), Lufts�ulen (Fl�te, Orgelpfeife, Panfl�te) oder St�be aus Holz oder Stahl (Xylophon) sein. Die Idee des Projekts war, eine Simulation zu erstellen um nachzuvollziehen wie Kl�nge entstehen. Die genaue Aufgabenstellung war am Anfang noch nicht klar und musste zuerst mit dem "Auftraggeber" Albert Heuberger definiert werden. 


\section{Aufgabenstellung / Abgrenzung}
Die Ausarbeitung der Aufgabenstellung hat zusammen mit Albert Heuberger stattgeefunden. Die Aufgabe dieser Arbeit besteht in der Simulation stehender Wellen, bzw. Schallwellen in R�hren. Die Simulation soll falls m�glich sowohl mit einem numerischen Verfahren, als auch analytisch m�glich sein.
Beispielsweise soll anhand einstellbarer L�nge und Durchmesser einer R�hre und dem darauf wirkenden Luftdruck ein Ton erzeugt werden.


\section{Abgrenzung}
In unserer Arbeit beschr�nken wir uns auf die Simulation von Schallwellen die in einem luftgef�llten Rohr entstehen. Sich in einem Rohr ausbreitende Wellen werden an beiden Enden des Rohres reflektiert und die Superposition der beiden entgegengesetzt fortlaufenden Wellen ergeben, sofern die Wellenl�nge  einer Resonanzfrequenz des Rohres entspricht, eine stehende Welle.


\section{M�gliche Erweiterungen}
-\ Simulation mit gebogenen R�hren \\
-\ Erzeugung spezieller Klangcharakteristika (z.B. verschiedener spezifischer Instrumente)


\section{Motivation}



\section{Projektverlauf}

\subsection{Iterationen}

\subsubsection{Iteration 1}
Termin: \\
\\
\emph{Ziele:} \\
-\ Erstellen des \emph{Proof of Concepts:} (Erzeugung von akustischen Signalen mit Java) \\
-\ Definieren des Projektrahmens mit dem Projektauftraggeber Albert Heuberger

\emph{Proof of Concept}
Die M�glichkeit mit Java akustische Signale zu erzeugen haben wir getestet mit der jsyn Library (erh�ltich unter\ \verb+http://www.softsynth.com/jsyn/index.php+)

\subsubsection{Iteration 2}
Termin: \\
\\
\emph{Ziele:} \\
-\ Ausarbeiten der physikalischen Grundlagen \\
-\ Verfahren zur Berechnung der Kl�nge ausw�hlen \\
-\ Erste Klassen definieren, Grundlagen des GUI erstellen

\subsubsection{Iteration 3}
Termin: \\
\\
\emph{Ziele:} \\
-\ physikalische Grundlagen detailliert verstehen \\
-\ Grundger�st der Dokumentation erstellen \\
-\ Formeln im Programm einbinden \\

\subsubsection{Iteration 4}
Termin: \\
\\
\emph{Ziele:} \\
-\ Dokumentation vervollst�ndigen \\
-\ Programm fertig stellen, erweitern

