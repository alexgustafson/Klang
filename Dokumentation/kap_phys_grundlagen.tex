\chapter{Physikalische Grundlagen}

\section{Stehende Wellen}
Wenn im Inneren einer R�hre eine stehende Welle erzeugt wird, entsteht ein Ton. Die Form und L�nge des Innenraums des Instruments beeinflusst die akustischen Wellen und somit Tonh�he und Klangfarbe.

\subsection{Was ist eine stehende Welle?}
Eine stehende Welle entsteht, wenn sich zwei gegenl�ufige, fortschreitende Wellen �berlagern. Die Wellen m�ssen die selbe Frequenz und Amplitude haben.
In einer R�hre entstehen diese Wellen dadurch, dass ein Impuls von Aussen eine Welle generiert welche dann an beiden Enden des Rohrs reflektiert wird.
Wie die Welle reflektiert wird h�ngt dabei davon ab, ob das Rohrende offen oder geschlossen ist. An einem offenen Rohrende befindet sich ein Schwingungsbauch (tats�chlich ist der Schwingungsbauch etwas ausserhalb des Rohrs, worauf wir aber hier nicht eingehen), an einem geschlossenen Rohrende befindet sich ein Knoten.

\subsubsection{Stehende Welle in einem an beiden Enden offenen Rohr}
Die stehende Welle, die in Abbildung 1 gezeigt wird, ist die einfachste stehende Welle in einem an beiden Enden offenen Rohr. Eine solche stehende Welle wird als \emph{Grundmode}, \emph{erste Schwingungsmode} oder \emph{1. Harmonische} bezeichnet. Damit die  \emph{1. Harmonische} entsteht muss f�r die Wellenl�nge $ \lambda $ und die L�nge $ L $ gelten:
$ \lambda=2L $ bzw. $ L=\lambda/2 $
Damit sich die \emph{zweite Schwingungsmode} ausbildet muss entsprechend die Schallwelle die Wellenl�nge $ \lambda=L $ betragen. F�r die \emph{3. Harmonische} muss die Wellenl�nge $ \lambda=2L/3 $ sein usw. \\
Allgemein gilt f�r ein an beiden Enden offenes Rohr der L�nge $ L $:\\
$ \lambda = \frac{2L}{n}$ , f�r $ n = 1,2,3,...$  wobei $ n $ als \emph{Modenzahl} bezeichnet wird.

\subsubsection{Stehende Welle in einem an einem Ende offenen Rohr}



\subsection{Entstehung einer Stehenden Welle}
Wird eine Sinuswelle reflektiert (am Ende der R�hre) ist die �berlagerung der urspr�nglichen und der reflektierten Sinuswelle die Stehende Welle. \\
Die stehende Welle ist also: Sinus Welle (nach rechhts) + reflektierte Sinus-Welle (nach links)


\begin{align*}
y &= A\ sin(kx - \omega t) + A\ sin(kx + \omega t)\\
  &= A\ sin(kx)\ cos(\omega t) - A\ cos(kx)\ sin(\omega t)\\
&\hspace{0.4cm}+ A\ sin(kx)\ cos(\omega t) + A\ cos(kx)\ sin(\omega t)\\
  &= 2A\ sin(kx) cos(\omega t)
\end{align*}

\section{Definitionen}



