\chapter{Programmierung}

\section{Grundstruktur}
Da das Generieren des Audiofiles sehr rechenintensiv ist, wird die Audiodatei jeweils gespeichert und nicht direkt ausgegeben. Das Audiofile kann kann nachdem es generiert wurde abgespielt werden.

\section{Programmiersprache}
Bei der Wahl der Programmiersprache hatten  die Vorgabe zu beachten, dass wir die Simulation mit einer Objektorientierten Programmiersprache realisieren. Da wir in der bisherigen Ausbildung haupts�chlich mit Java gearbeitet haben, haben wir uns auch bei diesem Projekt f�r Java entschieden.

\section{Komponenten}
benutzte Frameworks, Libraries

\section{Klassendiagramm}

\section{Code Erl�uterungen}


