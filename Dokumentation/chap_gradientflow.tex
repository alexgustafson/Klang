\chapter{Tangentialkegel und Differential}
\section{Kegel}
\begin{dfi}
Sei $(\Omega,\alpha)$ ein metrischer Raum und sei auf $[0,\infty)\times\Omega$ die �quivalenzrelation $\sim$ folgendermassen gegeben:
$$(r_1,\omega_1)\sim(r_2,\omega_2)\quad:\Longleftrightarrow\quad(r_1,\omega_1)=(r_2,\omega_2)\,\textrm{ oder }\,r_1=r_2=0$$
Wir definieren den \define{Kegel} �ber $(\Omega,\alpha)$ durch
$$C(\Omega):=[0,\infty)\times\Omega/\sim$$
und nennen die �quivalenzklasse $[0,\omega]=0$ \define{Kegelpunkt}. Wir definieren auf $C(\Omega)$ die Metrik $d$ durch
$$d([r_1,\omega_1],[r_2,\omega_2])^2=r_1^2+r_2^2-2r_1r_2\cos(\alpha_\pi(\omega_2,\omega_2)),$$
wobei
$$\alpha_\pi(\omega_1,\omega_2):=\min\{\alpha(\omega_1,\omega_2),\pi\}.$$
\end{dfi}
%
\graphic{kegel}
%
\begin{rem}
Die Metrik $d$ ist vom Cosinussatz f�r $\kappa=0$ abgeleitet.
\end{rem}
%
\begin{dfi}
Seien $[r_1,\omega_1]$ und $[r_2,\omega_2]$ in $C(\Omega)$. Wir definieren auf $(C(\Omega),d)$ die Norm $\|\,\|:C(\Omega)\rightarrow\R_{\geq0}$ durch $\|[r,\omega]\|:=r$, sowie das Skalarprodukt $\langle\,,\rangle:C(\Omega)\times C(\Omega)\rightarrow\R_{\geq0}$ durch
$$\langle[r_1,\omega_1],[r_2,\omega_2]\rangle:=r_1r_2\cos\big(\alpha_\pi(\omega_1,\omega_2)\big).$$
\end{dfi}
%
%
%
%
%
%
%
%
\section{Tangentialkegel}
In diesem Abschnitt bezeichne $(X,d)$ einen geod�tischen metrischen Raum, der $\mathrm{CBB}(\kappa)$ ist.
\begin{dfi}
Sei $x$ in $X$. Wir definieren f�r zwei K�rzeste $\sigma,\tau$, die in $x$ beginnen, die pseudo-Metrik
$$\alpha(\sigma,\tau):=\angle(\sigma,\tau),$$
sowie die �quivalenzrelation $\sim$ durch 
$$\sigma\sim\tau\quad\Longleftrightarrow\quad\alpha(\sigma,\tau)=0.$$
Somit k�nnen wir den metrischen Raum
$$\Omega_x':=\{\sigma\mid\sigma:[0,a]\rightarrow X\textrm{ K�rzeste mit }\sigma(0)=x\}/\sim.$$
mit der Metrik $\alpha$ definieren.\\
Wir nennen die Vervollst�ndigung $\Omega_x$ von $\Omega_x'$ den \define{Raum der Richtungen}. Falls $\Omega_x\neq\emptyset$, definieren den \define{Tangentialkegel} in $x$ durch $T_xX:=C(\Omega_x)$, andernfalls durch $T_xX:=\{0\}$.
\end{dfi}
%
\graphic{tangentialkegel}
%
