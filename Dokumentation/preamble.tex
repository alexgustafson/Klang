\documentclass[11pt,a4paper,twoside,draft,BCOR=1cm,DIV=11,headsepline]{scrreprt} %draft

%***************Start Paket-Einbindungen******************
\usepackage[latin1]{inputenc} %textcodierung, in windows auch latin1, in unix utf8
\usepackage[ngerman]{babel}		%deutsche Sprachtrennung
\usepackage{amsfonts,amsmath,amsthm}	
\usepackage{makeidx}	%fuer Verwendung des index'
\usepackage[colorlinks=false,pdfborder={0 0 0},plainpages=false]{hyperref}%fuer hyperlinks im pdf-format
%GENAU DANN verwenden, wenn digital veroeffentlicht
\usepackage{pgf} %fuer grafiken
%***************Ende Paket-Einbindungen******************

%**************Start Allgemeine Optionen*****************
\setkomafont{sectioning}{\rmfamily\bfseries\boldmath}
\makeindex
%\hyphenation{geo-d�-tisch Geo-d�-te L�n-gen-raum un-ter-halb-ste-tig un-ter-halb-ste-ti-ge Ver-gleichs-drei-ecke Alex-an-drov} %hier geh�ren zu trennende W�rter hin, die Latex nicht kennt
%% Gleichungen nach Kapitel nummerieren
\renewcommand{\theequation}{\thechapter.\arabic{equation}}
\numberwithin{equation}{chapter}
%\setlength{\parindent}{0pt} %kein Einzug (nie)
%\addtolength{\leftmargini}{2.5em}
%****************Ende Allgemeine Optionen*************

%************* Start Theoreme*************************
\theoremstyle{plain}
\newtheorem{theo}{Theorem}[chapter]
\newtheorem{lem}[theo]{Lemma}
\newtheorem{cor}[theo]{Korollar}
\newtheorem{pro}[theo]{Proposition}
\newtheorem{supp}[theo]{Vermutung}
\newtheorem{claim}[theo]{Behauptung}
\theoremstyle{definition}
\newtheorem{dfi}[theo]{Definition}
\newtheorem{con}[theo]{Konstruktion}
\theoremstyle{remark}
\newtheorem{rem}[theo]{Bemerkung}
\newtheorem{war}[theo]{Warnung}
\newtheorem{ntt}[theo]{Notation}
\newtheorem{exa}[theo]{Beispiel}
\newtheorem{exas}[theo]{Beispiele}
\newtheorem{exc}[theo]{\"Ubung}
%************** Ende Theoreme *************

%************** Start Befehle *************
%Aenderung der Nummerierungsart (klein roemisch)
\renewcommand{\labelenumi}{(\roman{enumi})}
%Das sind nur vereinfachungen
%bereits bestehender Befehle:
\providecommand{\define}[1]{\index{#1}\emph{#1}}
\providecommand{\Z}{\ensuremath{\mathbb{Z}}}
\providecommand{\N}{\ensuremath{\mathbb{N}}}
\providecommand{\C}{\ensuremath{\mathbb{C}}}
\providecommand{\Q}{\ensuremath{\mathbb{Q}}}
\providecommand{\R}{\ensuremath{\mathbb{R}}}
\providecommand{\de}{\ensuremath{\mathrm{d}}}
%************** Ende Befehle **************

%************** Start Operatoren **********
%das Bild 'im' willst du ja nicht kursiv
%gesetzt im mathemodus, deshalb ist es hier
%aufgefuehrt
\DeclareMathOperator{\id}{id}
\DeclareMathOperator{\im}{im}
\DeclareMathOperator{\sign}{sign}
%************** Ende Operatoren ***********

%************** Start Grafik **********
\providecommand{\graphic}[1]{\begin{center}{\footnotesize \input{graphics/#1.TpX}}\end{center}}
%************** Ende Grafik ***********
