\chapter{Physikalische Grundlagen}

\section{Stehende Wellen}
Wenn im Inneren einer R�hre eine stehende Welle erzeugt wird, entsteht ein Ton. Die Form und L�nge des Innenraums des Instruments beeinflusst die akustischen Wellen und somit Tonh�he und Klangfarbe.

\subsection{Was ist eine stehende Welle?}
Eine stehende Welle entsteht, wenn sich zwei gegenl�ufige, fortschreitende Wellen �berlagern. Die Wellen m�ssen die selbe Frequenz und Amplitude haben.
In einer R�hre entstehen diese Wellen durch Druck von Aussen und Reflektion (an einem geschlossen Ende) bzw. einen ausgleichenden Luftdruck (bei einem offenen Ende).


\subsection{Entstehung einer Stehenden Welle}
Wird eine Sinuswelle reflektiert (am Ende der R�hre) ist die �berlagerung der urspr�nglichen und der reflektierten Sinuswelle die Stehende Welle. \\
Die stehende Welle ist also: Sinus Welle (nach rechhts) + reflektierte Sinus-Welle (nach links)


\begin{align*}
y &= A\ sin(kx - \omega t) + A\ sin(kx + \omega t)\\
  &= A\ sin(kx)\ cos(\omega t) - A\ cos(kx)\ sin(\omega t)\\
&\hspace{0.4cm}+ A\ sin(kx)\ cos(\omega t) + A\ cos(kx)\ sin(\omega t)\\
  &= 2A\ sin(kx) cos(\omega t)
\end{align*}


